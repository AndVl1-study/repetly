% !TeX TXS-program:bibliography = txs:///biber
\documentclass[14pt, russian]{scrartcl}
\let\counterwithout\relax
\let\counterwithin\relax
%\usepackage{lmodern}
\usepackage{float}
\usepackage{xcolor}
\usepackage{extsizes}
\usepackage{subfig}
\usepackage[export]{adjustbox}
\usepackage{tocvsec2} % возможность менять учитываемую глубину разделов в оглавлении
\usepackage[subfigure]{tocloft}
\usepackage[newfloat]{minted}
\captionsetup[listing]{position=top}

\usepackage{fancyvrb}
\usepackage{ulem,bm,mathrsfs,ifsym} %зачеркивания, особо жирный стиль и RSFS начертание
\usepackage{sectsty} % переопределение стилей подразделов
%%%%%%%%%%%%%%%%%%%%%%%

%%% Поля и разметка страницы %%%
\usepackage{pdflscape}                              % Для включения альбомных страниц
\usepackage{geometry}                               % Для последующего задания полей
\geometry{a4paper,tmargin=2cm,bmargin=2cm,lmargin=3cm,rmargin=1cm} % тоже самое, но лучше

%%% Математические пакеты %%%
\usepackage{amsthm,amsfonts,amsmath,amssymb,amscd}  % Математические дополнения от AMS
\usepackage{mathtools}                              % Добавляет окружение multlined
\usepackage[perpage]{footmisc}
%\usepackage{times}

%%%% Установки для размера шрифта 14 pt %%%%
%% Формирование переменных и констант для сравнения (один раз для всех подключаемых файлов)%%
%% должно располагаться до вызова пакета fontspec или polyglossia, потому что они сбивают его работу
%\newlength{\curtextsize}
%\newlength{\bigtextsize}
%\setlength{\bigtextsize}{13pt}
\KOMAoptions{fontsize=14pt}

\makeatletter
\def\showfontsize{\f@size{} point}
\makeatother

%\makeatletter
%\show\f@size                                       % неплохо для отслеживания, но вызывает стопорение процесса, если документ компилируется без команды  -interaction=nonstopmode 
%\setlength{\curtextsize}{\f@size pt}
%\makeatother

%шрифт times
\usepackage{tempora} %только для тех, у кого MikTeX последней версии и не ловит pscyr
%\usepackage{pscyr} %для всех нормальных людей
%\setmainfont[Ligatures={TeX,Historic}]{Times New Roman}

   %%% Решение проблемы копирования текста в буфер кракозябрами
%    \input glyphtounicode.tex
%    \input glyphtounicode-cmr.tex %from pdfx package
%    \pdfgentounicode=1
    \usepackage{cmap}                               % Улучшенный поиск русских слов в полученном pdf-файле
    \usepackage[T1]{fontenc}                       % Поддержка русских букв
    \usepackage[utf8]{inputenc}                     % Кодировка utf8
    \usepackage[english, main=russian]{babel}            % Языки: русский, английский
%   \IfFileExists{pscyr.sty}{\usepackage{pscyr}}{}  % Красивые русские шрифты
%\renewcommand{\rmdefault}{ftm}
%%% Оформление абзацев %%%
\usepackage{indentfirst}                            % Красная строка
%\usepackage{eskdpz}

%%% Таблицы %%%
\usepackage{longtable}                              % Длинные таблицы
\usepackage{multirow,makecell,array}                % Улучшенное форматирование таблиц
\usepackage{booktabs}                               % Возможность оформления таблиц в классическом книжном стиле (при правильном использовании не противоречит ГОСТ)

%%% Общее форматирование
\usepackage{soulutf8}                               % Поддержка переносоустойчивых подчёркиваний и зачёркиваний
\usepackage{icomma}                                 % Запятая в десятичных дробях



%%% Изображения %%%
\usepackage{graphicx}                               % Подключаем пакет работы с графикой
\usepackage{wrapfig}

%%% Списки %%%
\usepackage{enumitem}

%%% Подписи %%%
\usepackage{caption}                                % Для управления подписями (рисунков и таблиц) % Может управлять номерами рисунков и таблиц с caption %Иногда может управлять заголовками в списках рисунков и таблиц
%% Использование:
%\begin{table}[h!]\ContinuedFloat - чтобы не переключать счетчик
%\captionsetup{labelformat=continued}% должен стоять до самого caption
%\caption{}
% либо ручками \caption*{Продолжение таблицы~\ref{...}.} :)

%%% Интервалы %%%
\addto\captionsrussian{%
  \renewcommand{\listingname}{Листинг}%
}
%%% Счётчики %%%
\usepackage[figure,table,section]{totalcount}               % Счётчик рисунков и таблиц
\DeclareTotalCounter{lstlisting}
\usepackage{totcount}                               % Пакет создания счётчиков на основе последнего номера подсчитываемого элемента (может требовать дважды компилировать документ)
\usepackage{totpages}                               % Счётчик страниц, совместимый с hyperref (ссылается на номер последней страницы). Желательно ставить последним пакетом в преамбуле

%%% Продвинутое управление групповыми ссылками (пока только формулами) %%%
%% Кодировки и шрифты %%%

%   \newfontfamily{\cyrillicfont}{Times New Roman}
%   \newfontfamily{\cyrillicfonttt}{CMU Typewriter Text}
	%\setmainfont{Times New Roman}
	%\newfontfamily\cyrillicfont{Times New Roman}
	%\setsansfont{Times New Roman}                    %% задаёт шрифт без засечек
%	\setmonofont{Liberation Mono}               %% задаёт моноширинный шрифт
%    \IfFileExists{pscyr.sty}{\renewcommand{\rmdefault}{ftm}}{}
%%% Интервалы %%%
%linespread-реализация ближе к реализации полуторного интервала в ворде.
%setspace реализация заточена под шрифты 10, 11, 12pt, под остальные кегли хуже, но всё же ближе к типографской классике. 
\linespread{1.3}                    % Полуторный интервал (ГОСТ Р 7.0.11-2011, 5.3.6)
%\renewcommand{\@biblabel}[1]{#1}

%%% Гиперссылки %%%
\usepackage{hyperref}

%%% Выравнивание и переносы %%%
\sloppy                             % Избавляемся от переполнений
\clubpenalty=10000                  % Запрещаем разрыв страницы после первой строки абзаца
\widowpenalty=10000                 % Запрещаем разрыв страницы после последней строки абзаца

\makeatletter % малые заглавные, small caps shape
\let\@@scshape=\scshape
\renewcommand{\scshape}{%
  \ifnum\strcmp{\f@series}{bx}=\z@
    \usefont{T1}{cmr}{bx}{sc}%
  \else
    \ifnum\strcmp{\f@shape}{it}=\z@
      \fontshape{scsl}\selectfont
    \else
      \@@scshape
    \fi
  \fi}
\makeatother

%%% Подписи %%%
%\captionsetup{%
%singlelinecheck=off,                % Многострочные подписи, например у таблиц
%skip=2pt,                           % Вертикальная отбивка между подписью и содержимым рисунка или таблицы определяется ключом
%justification=centering,            % Центрирование подписей, заданных командой \caption
%}
%%%        Подключение пакетов                 %%%
\usepackage{ifthen}                 % добавляет ifthenelse
%%% Инициализирование переменных, не трогать!  %%%
\newcounter{intvl}
\newcounter{otstup}
\newcounter{contnumeq}
\newcounter{contnumfig}
\newcounter{contnumtab}
\newcounter{pgnum}
\newcounter{bibliosel}
\newcounter{chapstyle}
\newcounter{headingdelim}
\newcounter{headingalign}
\newcounter{headingsize}
\newcounter{tabcap}
\newcounter{tablaba}
\newcounter{tabtita}
%%%%%%%%%%%%%%%%%%%%%%%%%%%%%%%%%%%%%%%%%%%%%%%%%%

%%% Область упрощённого управления оформлением %%%

%% Интервал между заголовками и между заголовком и текстом
% Заголовки отделяют от текста сверху и снизу тремя интервалами (ГОСТ Р 7.0.11-2011, 5.3.5)
\setcounter{intvl}{3}               % Коэффициент кратности к размеру шрифта

%% Отступы у заголовков в тексте
\setcounter{otstup}{0}              % 0 --- без отступа; 1 --- абзацный отступ

%% Нумерация формул, таблиц и рисунков
\setcounter{contnumeq}{1}           % Нумерация формул: 0 --- пораздельно (во введении подряд, без номера раздела); 1 --- сквозная нумерация по всей диссертации
\setcounter{contnumfig}{1}          % Нумерация рисунков: 0 --- пораздельно (во введении подряд, без номера раздела); 1 --- сквозная нумерация по всей диссертации
\setcounter{contnumtab}{1}          % Нумерация таблиц: 0 --- пораздельно (во введении подряд, без номера раздела); 1 --- сквозная нумерация по всей диссертации

%% Оглавление
\setcounter{pgnum}{0}               % 0 --- номера страниц никак не обозначены; 1 --- Стр. над номерами страниц (дважды компилировать после изменения)

%% Библиография
\setcounter{bibliosel}{1}           % 0 --- встроенная реализация с загрузкой файла через движок bibtex8; 1 --- реализация пакетом biblatex через движок biber

%% Текст и форматирование заголовков
\setcounter{chapstyle}{1}           % 0 --- разделы только под номером; 1 --- разделы с названием "Глава" перед номером
\setcounter{headingdelim}{1}        % 0 --- номер отделен пропуском в 1em или \quad; 1 --- номера разделов и приложений отделены точкой с пробелом, подразделы пропуском без точки; 2 --- номера разделов, подразделов и приложений отделены точкой с пробелом.

%% Выравнивание заголовков в тексте
\setcounter{headingalign}{0}        % 0 --- по центру; 1 --- по левому краю

%% Размеры заголовков в тексте
\setcounter{headingsize}{0}         % 0 --- по ГОСТ, все всегда 14 пт; 1 --- пропорционально изменяющийся размер в зависимости от базового шрифта

%% Подпись таблиц
\setcounter{tabcap}{0}              % 0 --- по ГОСТ, номер таблицы и название разделены тире, выровнены по левому краю, при необходимости на нескольких строках; 1 --- подпись таблицы не по ГОСТ, на двух и более строках, дальнейшие настройки: 
%Выравнивание первой строки, с подписью и номером
\setcounter{tablaba}{2}             % 0 --- по левому краю; 1 --- по центру; 2 --- по правому краю
%Выравнивание строк с самим названием таблицы
\setcounter{tabtita}{1}             % 0 --- по левому краю; 1 --- по центру; 2 --- по правому краю

%%% Рисунки %%%
\DeclareCaptionLabelSeparator*{emdash}{~--- }             % (ГОСТ 2.105, 4.3.1)
\captionsetup[figure]{labelsep=emdash,font=onehalfspacing,position=bottom}

%%% Таблицы %%%
\ifthenelse{\equal{\thetabcap}{0}}{%
    \newcommand{\tabcapalign}{\raggedright}  % по левому краю страницы или аналога parbox
}

\ifthenelse{\equal{\thetablaba}{0} \AND \equal{\thetabcap}{1}}{%
    \newcommand{\tabcapalign}{\raggedright}  % по левому краю страницы или аналога parbox
}

\ifthenelse{\equal{\thetablaba}{1} \AND \equal{\thetabcap}{1}}{%
    \newcommand{\tabcapalign}{\centering}    % по центру страницы или аналога parbox
}

\ifthenelse{\equal{\thetablaba}{2} \AND \equal{\thetabcap}{1}}{%
    \newcommand{\tabcapalign}{\raggedleft}   % по правому краю страницы или аналога parbox
}

\ifthenelse{\equal{\thetabtita}{0} \AND \equal{\thetabcap}{1}}{%
    \newcommand{\tabtitalign}{\raggedright}  % по левому краю страницы или аналога parbox
}

\ifthenelse{\equal{\thetabtita}{1} \AND \equal{\thetabcap}{1}}{%
    \newcommand{\tabtitalign}{\centering}    % по центру страницы или аналога parbox
}

\ifthenelse{\equal{\thetabtita}{2} \AND \equal{\thetabcap}{1}}{%
    \newcommand{\tabtitalign}{\raggedleft}   % по правому краю страницы или аналога parbox
}

\DeclareCaptionFormat{tablenocaption}{\tabcapalign #1\strut}        % Наименование таблицы отсутствует
\ifthenelse{\equal{\thetabcap}{0}}{%
    \DeclareCaptionFormat{tablecaption}{\tabcapalign #1#2#3}
    \captionsetup[table]{labelsep=emdash}                       % тире как разделитель идентификатора с номером от наименования
}{%
    \DeclareCaptionFormat{tablecaption}{\tabcapalign #1#2\par%  % Идентификатор таблицы на отдельной строке
        \tabtitalign{#3}}                                       % Наименование таблицы строкой ниже
    \captionsetup[table]{labelsep=space}                        % пробельный разделитель идентификатора с номером от наименования
}
\captionsetup[table]{format=tablecaption,singlelinecheck=off,font=onehalfspacing,position=top,skip=-5pt}  % многострочные наименования и прочее
\DeclareCaptionLabelFormat{continued}{Продолжение таблицы~#2}
\setlength{\belowcaptionskip}{.2cm}
\setlength{\intextsep}{0ex}

%%% Подписи подрисунков %%%
\renewcommand{\thesubfigure}{\asbuk{subfigure}}           % Буквенные номера подрисунков
\captionsetup[subfigure]{font={normalsize},               % Шрифт подписи названий подрисунков (не отличается от основного)
    labelformat=brace,                                    % Формат обозначения подрисунка
    justification=centering,                              % Выключка подписей (форматирование), один из вариантов            
}
%\DeclareCaptionFont{font12pt}{\fontsize{12pt}{13pt}\selectfont} % объявляем шрифт 12pt для использования в подписях, тут же надо интерлиньяж объявлять, если не наследуется
%\captionsetup[subfigure]{font={font12pt}}                 % Шрифт подписи названий подрисунков (всегда 12pt)

%%% Настройки гиперссылок %%%

\definecolor{linkcolor}{rgb}{0.0,0,0}
\definecolor{citecolor}{rgb}{0,0.0,0}
\definecolor{urlcolor}{rgb}{0,0,0}

\hypersetup{
    linktocpage=true,           % ссылки с номера страницы в оглавлении, списке таблиц и списке рисунков
%    linktoc=all,                % both the section and page part are links
%    pdfpagelabels=false,        % set PDF page labels (true|false)
    plainpages=true,           % Forces page anchors to be named by the Arabic form  of the page number, rather than the formatted form
    colorlinks,                 % ссылки отображаются раскрашенным текстом, а не раскрашенным прямоугольником, вокруг текста
    linkcolor={linkcolor},      % цвет ссылок типа ref, eqref и подобных
    citecolor={citecolor},      % цвет ссылок-цитат
    urlcolor={urlcolor},        % цвет гиперссылок
    pdflang={ru},
}
\urlstyle{same}
%%% Шаблон %%%
%\DeclareRobustCommand{\todo}{\textcolor{red}}       % решаем проблему превращения названия цвета в результате \MakeUppercase, http://tex.stackexchange.com/a/187930/79756 , \DeclareRobustCommand protects \todo from expanding inside \MakeUppercase
\setlength{\parindent}{2.5em}                       % Абзацный отступ. Должен быть одинаковым по всему тексту и равен пяти знакам (ГОСТ Р 7.0.11-2011, 5.3.7).

%%% Списки %%%
% Используем дефис для ненумерованных списков (ГОСТ 2.105-95, 4.1.7)
%\renewcommand{\labelitemi}{\normalfont\bfseries~{---}} 
\renewcommand{\labelitemi}{\bfseries~{---}} 
\setlist{nosep,%                                    % Единый стиль для всех списков (пакет enumitem), без дополнительных интервалов.
    labelindent=\parindent,leftmargin=*%            % Каждый пункт, подпункт и перечисление записывают с абзацного отступа (ГОСТ 2.105-95, 4.1.8)
}
%%%%%%%%%%%%%%%%%%%%%%
%\usepackage{xltxtra} % load xunicode

\usepackage{ragged2e}
\usepackage[explicit]{titlesec}
\usepackage{placeins}
\usepackage{xparse}
\usepackage{csquotes}

\usepackage{listingsutf8}
\usepackage{url} %пакеты расширений
\usepackage{algorithm, algorithmicx}
\usepackage[noend]{algpseudocode}
\usepackage{blkarray}
\usepackage{chngcntr}
\usepackage{tabularx}
\usepackage[backend=biber, 
    bibstyle=gost-numeric,
    citestyle=nature]{biblatex}
\newcommand*\template[1]{\text{<}#1\text{>}}
\addbibresource{biblio.bib}
  
\titleformat{name=\section,numberless}[block]{\normalfont\large\centering}{}{0em}{#1}
\titleformat{\section}[block]{\normalfont\large\bfseries\raggedright}{}{0em}{\thesection\hspace{0.25em}#1}
\titleformat{\subsection}[block]{\normalfont\large\bfseries\raggedright}{}{0em}{\thesubsection\hspace{0.25em}#1}
\titleformat{\subsubsection}[block]{\normalfont\large\bfseries\raggedright}{}{0em}{\thesubsubsection\hspace{0.25em}#1}

\let\Algorithm\algorithm
\renewcommand\algorithm[1][]{\Algorithm[#1]\setstretch{1.5}}
%\renewcommand{\listingscaption}{Листинг}

\usepackage{pifont}
\usepackage{calc}
\usepackage{suffix}
\usepackage{csquotes}
\DeclareQuoteStyle{russian}
    {\guillemotleft}{\guillemotright}[0.025em]
    {\quotedblbase}{\textquotedblleft}
\ExecuteQuoteOptions{style=russian}
\newcommand{\enq}[1]{\enquote{#1}}  
\newcommand{\eng}[1]{\begin{english}#1\end{english}}
% Подчиненные счетчики в окружениях http://old.kpfu.ru/journals/izv_vuz/arch/sample1251.tex
\newcounter{cTheorem} 
\newcounter{cDefinition}
\newcounter{cConsequent}
\newcounter{cExample}
\newcounter{cLemma}
\newcounter{cConjecture}
\newtheorem{Theorem}{Теорема}[cTheorem]
\newtheorem{Definition}{Определение}[cDefinition]
\newtheorem{Consequent}{Следствие}[cConsequent]
\newtheorem{Example}{Пример}[cExample]
\newtheorem{Lemma}{Лемма}[cLemma]
\newtheorem{Conjecture}{Гипотеза}[cConjecture]

\renewcommand{\theTheorem}{\arabic{Theorem}}
\renewcommand{\theDefinition}{\arabic{Definition}}
\renewcommand{\theConsequent}{\arabic{Consequent}}
\renewcommand{\theExample}{\arabic{Example}}
\renewcommand{\theLemma}{\arabic{Lemma}}
\renewcommand{\theConjecture}{\arabic{Conjecture}}
%\makeatletter
\NewDocumentCommand{\Newline}{}{\text{\\}}
\newcommand{\sequence}[2]{\ensuremath \left(#1,\ \dots,\ #2\right)}

\definecolor{mygreen}{rgb}{0,0.6,0}
\definecolor{mygray}{rgb}{0.5,0.5,0.5}
\definecolor{mymauve}{rgb}{0.58,0,0.82}
\renewcommand{\listalgorithmname}{Список алгоритмов}
\floatname{algorithm}{Листинг}
\renewcommand{\lstlistingname}{Листинг}
\renewcommand{\thealgorithm}{\arabic{algorithm}}

\newcommand{\refAlgo}[1]{(листинг \ref{#1})}
\newcommand{\refImage}[1]{(рисунок \ref{#1})}

\renewcommand{\theenumi}{\arabic{enumi}.}% Меняем везде перечисления на цифра.цифра	
\renewcommand{\labelenumi}{\arabic{enumi}.}% Меняем везде перечисления на цифра.цифра
\renewcommand{\theenumii}{\arabic{enumii}}% Меняем везде перечисления на цифра.цифра
\renewcommand{\labelenumii}{(\arabic{enumii})}% Меняем везде перечисления на цифра.цифра
\renewcommand{\theenumiii}{\roman{enumiii}}% Меняем везде перечисления на цифра.цифра
\renewcommand{\labelenumiii}{(\roman{enumiii})}% Меняем везде перечисления на цифра.цифра
%\newfontfamily\AnkaCoder[Path=src/fonts/]{AnkaCoder-r.ttf}
\renewcommand{\labelitemi}{---}
\renewcommand{\labelitemii}{---}

%\usepackage{courier}

\makeatletter
\def\p@subsection{}
\def\p@subsubsection{\thesection\,\thesubsection\,}
\makeatother
\newcommand{\prog}[1]{{\ttfamily\small#1}}

\newcommand{\anonsection}[1]{\cleardoublepage
\phantomsection
\addcontentsline{toc}{section}{\protect\numberline{}#1}
\section*{#1}\vspace*{2.5ex} % По госту положены 3 пустые строки после заголовка ненумеруемого раздела
}
\newcommand{\sectionbreak}{\clearpage}
\renewcommand{\sectionfont}{\normalsize} % Сбиваем стиль оглавления в стандартный
\renewcommand{\cftsecleader}{\cftdotfill{\cftdotsep}} % Точки в оглавлении напротив разделов

\renewcommand{\cftsecfont}{\normalfont\large} % Переключение на times в содержании
\renewcommand{\cftsubsecfont}{\normalfont\large} % Переключение на times в содержании

\usepackage{caption} 
%\captionsetup[table]{justification=raggedleft} 
%\captionsetup[figure]{justification=centering,labelsep=endash}
\usepackage{amsmath}    % \bar    (матрицы и проч. ...)
\usepackage{amsfonts}   % \mathbb (символ для множества действительных чисел и проч. ...)
\usepackage{mathtools}  % \abs, \norm
    \DeclarePairedDelimiter\abs{\lvert}{\rvert} % операция модуля
    \DeclarePairedDelimiter\norm{\lVert}{\rVert} % операция нормы
\DeclareTextCommandDefault{\textvisiblespace}{%
  \mbox{\kern.06em\vrule \@height.3ex}%
  \vbox{\hrule \@width.3em}%
  \hbox{\vrule \@height.3ex}}    
\newsavebox{\spacebox}
\begin{lrbox}{\spacebox}
\verb*! !
\end{lrbox}
\newcommand{\aspace}{\usebox{\spacebox}}
\DeclareTotalCounter{listing}

\makeatletter
\renewcommand*{\p@subsubsection}{}
\makeatother

\newenvironment{longlisting}{\captionsetup{type=listing}}{}

\begin{document}
\sloppy

\begin{titlepage}
\begin{center}

{\large \textbf{ФЕДЕРАЛЬНОЕ ГОСУДАРСТВЕННОЕ АВТОНОМНОЕ}}\\[0.3cm]
{\large \textbf{ОБРАЗОВАТЕЛЬНОЕ УЧРЕЖДЕНИЕ ВЫСШЕГО ОБРАЗОВАНИЯ}}\\[0.5cm]

\includegraphics[width=0.2\textwidth]{emblem.png}\\[0.5cm]

{\large \textbf{МОСКОВСКИЙ ПОЛИТЕХНИЧЕСКИЙ УНИВЕРСИТЕТ}}\\[0.5cm]

{\large Факультет информационных технологий}\\[0.3cm]
{\large Кафедра Информатики и информационных технологий}\\[1cm]

{\large направление подготовки 09.04.02 «Информационные системы и}\\
{\large технологии»,}\\[0.3cm]
{\large профиль «Мобильные технологии»}\\[2cm]

{\Large \textbf{КУРСОВАЯ РАБОТА}}\\[1cm]

{\large \textbf{Дисциплина:} «Мобильные операционные системы»}\\[3cm]

\end{center}

\begin{flushright}
\begin{tabular}{ll}
{\large \textbf{Выполнил:} студент группы 211-362}\\
{\large Владиславов А.А.}\\[0.5cm]
{\large \textbf{Дата, подпись} \rule{5cm}{0.4pt} \rule{5cm}{0.4pt}}\\
{\scriptsize \hspace*{7.5cm}(Дата) \hspace{3.5cm}(Подпись)}\\[1cm]
{\large \textbf{Проверил:} к.т.н., доцент}\\
{\large Кузнецов И.А.}\\[0.5cm]
{\large \textbf{Дата, подпись} \rule{5cm}{0.4pt} \rule{5cm}{0.4pt}}\\
{\scriptsize \hspace*{7.5cm}(Дата) \hspace{3.5cm}(Подпись)}\\[0.3cm]
{\large \hspace*{11cm}(Оценка)}
\end{tabular}
\end{flushright}

\vspace{1cm}
{\large \textbf{Замечания:}}\\[0.3cm]
\rule{\textwidth}{0.4pt}\\[0.3cm]
\rule{\textwidth}{0.4pt}

\vfill

\begin{center}
{\large \textbf{Москва}}\\
{\large \textbf{2024}}
\end{center}

\end{titlepage}
%\renewcommand{\ttdefault}{pcr}

\setlength{\tabcolsep}{3pt}
\newpage
\setcounter{page}{2}
%----------------------------------------------------------------------------
%                  ОТСЮДА --- СОБСТВЕННО ТЕКСТ
%----------------------------------------------------------------------------
\renewcommand\contentsname{\hfill{\normalfont{СОДЕРЖАНИЕ}}\hfill}  %Оглавление
\tableofcontents
\newpage
\anonsection{ВВЕДЕНИЕ}  %Введение

Введение

\section{Разработка базового мобильного приложения}\label{sect:development}

\subsection{Техническое задание}\label{sect:requirements}

Цель разработки проекта – создание мобильного приложения для платформы Android, предназначенного для поиска и взаимодействия между репетиторами и учениками. Приложение призвано упростить процесс поиска подходящего репетитора или ученика, организации занятий и отслеживания прогресса обучения.

Основные задачи проекта:

1. Реализация системы профилей пользователей.
   \begin{itemize}
   \item Создание профиля репетитора с указанием предметов, опыта, стоимости занятий
   \item Создание профиля ученика с указанием интересующих предметов и целей обучения
   \item Возможность добавления портфолио и отзывов
   \end{itemize}

2. Разработка системы поиска и подбора.
   \begin{itemize}
   \item Поиск репетиторов по предметам, стоимости, рейтингу
   \item Поиск учеников по предметам и целям обучения
   \item Фильтрация по местоположению и формату занятий (онлайн/офлайн)
   \item Система рекомендаций на основе предпочтений пользователя
   \end{itemize}

3. Организация взаимодействия между пользователями.
   \begin{itemize}
   \item Система сообщений для общения
   \item Планирование и управление расписанием занятий
   \item Отслеживание прогресса обучения
   \item Система оценок и отзывов
   \end{itemize}

4. Обеспечение безопасности и надежности.
   \begin{itemize}
   \item Верификация профилей пользователей
   \item Защита персональных данных
   \item Безопасная система оплаты
   \item Разрешение спорных ситуаций
   \end{itemize}

Эти функции делают приложение полезным инструментом для всех, кто стремится к эффективному управлению своим временем и задачами.

В рамках первого этапа было разработано мобильное приложение для управления повторяющимися задачами (Repetly). Приложение позволяет пользователям создавать, отслеживать и управлять регулярными задачами, получая уведомления о необходимости их выполнения.

\subsection{Архитектура приложения}\label{sect:architecture}

Приложение разработано с использованием современного стека технологий Android и следует принципам Clean Architecture. Основные технологические решения включают:

\begin{itemize}
\item Язык программирования Kotlin
\item Архитектурный паттерн MVVM (Model-View-ViewModel)
\item Jetpack Compose для построения пользовательского интерфейса
\item Room для работы с локальной базой данных
\item Hilt для внедрения зависимостей
\item WorkManager для управления фоновыми задачами
\item Firebase для аутентификации и облачного хранения
\end{itemize}

\subsection{Реализация базы данных}\label{sect:database}

Для работы с локальными данными используется Room. Архитектура базы данных построена с учетом необходимости кэширования и периодической очистки устаревших данных.

\begin{listing}[!htb]
\caption{Определение DAO для работы с пользователями}
\label{lst:user-dao}
\begin{minted}[frame=single,fontsize = \footnotesize, linenos, xleftmargin = 1.5em]{kotlin}
@Dao
interface UserDao {
    @Query("SELECT * FROM users WHERE uid = :uid")
    suspend fun getUserById(uid: String): UserEntity?

    @Insert(onConflict = OnConflictStrategy.REPLACE)
    suspend fun insertUser(user: UserEntity)

    @Query("DELETE FROM users WHERE createdAt < :timestamp")
    suspend fun deleteOldUsers(timestamp: Long)
}
\end{minted}
\end{listing}

Для обеспечения эффективного использования памяти устройства реализован механизм автоматической очистки устаревших данных с использованием WorkManager (см. листинг~\ref{lst:worker}). Этот подход позволяет надежно планировать и выполнять фоновые задачи, которые продолжат работать даже после перезапуска устройства. Планирование очистки кэша осуществляется через специальный сервис (см. листинг~\ref{lst:cache-service}), который запускается при старте приложения.

\begin{listing}[!htb]
\caption{Сервис управления кэшем}
\label{lst:cache-service}
\begin{minted}[frame=single,fontsize = \footnotesize, linenos, xleftmargin = 1.5em]{kotlin}
@Singleton
class CacheMaintenanceService @Inject constructor(
    private val workManager: WorkManager
) {
    fun scheduleCacheCleanup() {
        val cleanupRequest = PeriodicWorkRequestBuilder<CacheCleanupWorker>(
            repeatInterval = 1,
            repeatIntervalTimeUnit = TimeUnit.DAYS
        ).build()

        workManager.enqueueUniquePeriodicWork(
            CACHE_CLEANUP_WORK_NAME,
            ExistingPeriodicWorkPolicy.KEEP,
            cleanupRequest
        )
    }

    companion object {
        private const val CACHE_CLEANUP_WORK_NAME = "cache_cleanup_work"
    }
}
\end{minted}
\end{listing}

\begin{listing}[!htb]
\caption{Реализация Worker'а для очистки кэша}
\label{lst:worker}
\begin{minted}[frame=single,fontsize = \footnotesize, linenos, xleftmargin = 1.5em]{kotlin}
@HiltWorker
class CacheCleanupWorker @AssistedInject constructor(
    @Assisted appContext: Context,
    @Assisted workerParams: WorkerParameters,
    private val userDao: UserDao
) : CoroutineWorker(appContext, workerParams) {

    override suspend fun doWork(): Result {
        return try {
            val oldTimestamp = System.currentTimeMillis() - CACHE_EXPIRATION
            userDao.deleteOldUsers(oldTimestamp)
            Result.success()
        } catch (e: Exception) {
            Result.retry()
        }
    }

    companion object {
        private const val CACHE_EXPIRATION = 7L * 24 * 60 * 60 * 1000 // 7 дней
    }
}
\end{minted}
\end{listing}

\subsection{Система уведомлений и обратной связи}\label{sect:notifications}

В рамках разработки приложения было принято решение использовать встроенные механизмы обратной связи вместо системных push-уведомлений. Это решение обусловлено тем, что Firebase Cloud Messaging, который изначально планировался для отправки уведомлений, больше не поддерживает выполнение пользовательских скриптов для отправки push-уведомлений в бесплатном режиме.

Вместо этого в приложении реализована система внутренних уведомлений, включающая:

\begin{itemize}
\item Snackbar для кратковременных сообщений об успешных действиях
\item Dialog для важных сообщений, требующих внимания пользователя
\item BottomSheet для отображения дополнительной информации
\end{itemize}

Пример реализации компонента для отображения сообщений (см. листинг~\ref{lst:snackbar}):

\begin{listing}[!htb]
\caption{Реализация компонента для отображения Snackbar}
\label{lst:snackbar}
\begin{minted}[frame=single,fontsize = \footnotesize, linenos, xleftmargin = 1.5em]{kotlin}
@Composable
fun MessageHost(
    snackbarHostState: SnackbarHostState,
    modifier: Modifier = Modifier
) {
    SnackbarHost(
        hostState = snackbarHostState,
        modifier = modifier,
        snackbar = { data ->
            Snackbar(
                modifier = Modifier.padding(16.dp),
                content = {
                    Text(
                        text = data.message,
                        style = MaterialTheme.typography.bodyMedium
                    )
                },
                action = data.actionLabel?.let {
                    {
                        TextButton(onClick = { data.performAction() }) {
                            Text(text = data.actionLabel)
                        }
                    }
                }
            )
        }
    )
}
\end{minted}
\end{listing}

Для управления фоновыми задачами и периодической очистки кэша используется WorkManager (см. листинг~\ref{lst:worker}), который обеспечивает надежное выполнение запланированных операций даже после перезапуска устройства.

\subsection{Пользовательский интерфейс}\label{sect:ui}

Пользовательский интерфейс реализован с использованием современного инструментария Jetpack Compose, который представляет собой декларативный фреймворк для создания пользовательских интерфейсов. По сравнению с традиционным подходом на основе View, Jetpack Compose предоставляет ряд существенных преимуществ:

\begin{itemize}
\item \textbf{Меньше кода}: Compose позволяет создавать сложные интерфейсы с меньшим количеством кода по сравнению с XML-разметкой. Например, создание списка с пользовательской анимацией в XML потребовало бы нескольких файлов разметки и адаптеров, тогда как в Compose это можно сделать в одной функции.

\item \textbf{Типобезопасность}: В отличие от XML, где ошибки обнаруживаются только во время выполнения, Compose обеспечивает проверку типов на этапе компиляции.

\item \textbf{Встроенный предпросмотр}: Возможность видеть результат изменений UI в реальном времени без запуска приложения через аннотацию @Preview.

\item \textbf{Улучшенная производительность}: Compose использует эффективный механизм перерисовки, обновляя только те компоненты, которые действительно изменились.

\item \textbf{Легкость тестирования}: Компоненты Compose легко тестировать благодаря их функциональной природе и предсказуемому поведению.
\end{itemize}

Основные экраны приложения включают:

\begin{itemize}
\item Экран авторизации
\item Главный экран со списком задач
\item Экран создания/редактирования задачи
\item Экран настроек
\end{itemize}

\textbf{Здесь нужен скриншот главного экрана приложения}

Для навигации между экранами используется Compose Navigation (см. листинг~\ref{lst:navigation}):

\begin{listing}[!htb]
\caption{Определение навигации в приложении}
\label{lst:navigation}
\begin{minted}[frame=single,fontsize = \footnotesize, linenos, xleftmargin = 1.5em]{kotlin}
@Composable
fun AppNavigation() {
    val navController = rememberNavController()
    
    NavHost(
        navController = navController,
        startDestination = "auth"
    ) {
        composable("auth") { AuthScreen(navController) }
        composable("main") { MainScreen(navController) }
        composable("task/{taskId}") { backStackEntry ->
            val taskId = backStackEntry.arguments?.getString("taskId")
            TaskScreen(taskId, navController)
        }
        composable("settings") { SettingsScreen(navController) }
    }
}
\end{minted}
\end{listing}

\subsection{Внедрение зависимостей}\label{sect:di}

Для управления зависимостями в приложении используется Hilt - библиотека для внедрения зависимостей, построенная на основе Dagger. Выбор Hilt обусловлен несколькими факторами:

\begin{itemize}
\item По сравнению с базовым Dagger 2:
  \begin{itemize}
  \item Значительно упрощенный синтаксис
  \item Предопределенные скоупы для Android компонентов
  \item Меньше шаблонного кода
  \item Встроенная поддержка WorkManager и Navigation
  \end{itemize}
\item По сравнению с Koin:
  \begin{itemize}
  \item Проверка зависимостей на этапе компиляции
  \item Лучшая производительность благодаря кодогенерации
  \item Более надежная работа с большими проектами
  \item Официальная поддержка от Google
  \end{itemize}
\end{itemize}

Рассмотрим основные компоненты системы внедрения зависимостей. В листинге~\ref{lst:hilt-application} показана реализация базового класса приложения:

\begin{listing}[!htb]
\caption{Определение Application класса с поддержкой Hilt}
\label{lst:hilt-application}
\begin{minted}[frame=single,fontsize = \footnotesize, linenos, xleftmargin = 1.5em]{kotlin}
@HiltAndroidApp
class RepetlyApplication : Application() {
    @Inject
    lateinit var cacheMaintenanceService: CacheMaintenanceService

    override fun onCreate() {
        super.onCreate()
        cacheMaintenanceService.scheduleCacheCleanup()
    }
}
\end{minted}
\end{listing}

Для различных компонентов приложения созданы специализированные модули. Пример модуля для работы с аутентификацией показан в листинге~\ref{lst:auth-module}:

\begin{listing}[!htb]
\caption{Модуль для работы с Firebase Authentication}
\label{lst:auth-module}
\begin{minted}[frame=single,fontsize = \footnotesize, linenos, xleftmargin = 1.5em]{kotlin}
@Module
@InstallIn(SingletonComponent::class)
object AuthModule {
    @Provides
    @Singleton
    fun provideFirebaseAuth(): FirebaseAuth = FirebaseAuth.getInstance()

    @Provides
    @Singleton
    fun provideAuthRepository(
        firebaseAuth: FirebaseAuth,
        userDao: UserDao
    ): AuthRepository = FirebaseAuthRepository(firebaseAuth, userDao)
}
\end{minted}
\end{listing}

Для управления фоновыми задачами создан специальный модуль инициализации WorkManager (см. листинг~\ref{lst:worker-module}):

\begin{listing}[!htb]
\caption{Модуль инициализации WorkManager}
\label{lst:worker-module}
\begin{minted}[frame=single,fontsize = \footnotesize, linenos, xleftmargin = 1.5em]{kotlin}
@Module
@InstallIn(SingletonComponent::class)
object WorkManagerInitializer {
    @Provides
    @Singleton
    fun provideWorkManagerConfiguration(
        workerFactory: HiltWorkerFactory
    ): Configuration {
        return Configuration.Builder()
            .setWorkerFactory(workerFactory)
            .setMinimumLoggingLevel(android.util.Log.INFO)
            .build()
    }

    @Provides
    @Singleton
    fun provideWorkManager(
        @ApplicationContext context: Context,
        configuration: Configuration
    ): WorkManager {
        WorkManager.initialize(context, configuration)
        return WorkManager.getInstance(context)
    }
}
\end{minted}
\end{listing}

Такая организация кода позволяет:
\begin{itemize}
\item Легко тестировать компоненты благодаря возможности подмены зависимостей
\item Избежать сильной связанности между модулями
\item Централизованно управлять жизненным циклом объектов
\item Автоматически внедрять зависимости в Android компоненты
\end{itemize}

\subsection{Результаты разработки}\label{sect:results}

В результате первого этапа разработки было создано базовое мобильное приложение, которое включает:

\begin{itemize}
\item Систему аутентификации пользователей
\item Локальное хранение данных с использованием Room
\item Систему уведомлений на основе WorkManager
\item Современный пользовательский интерфейс на Jetpack Compose
\item Архитектуру, основанную на лучших практиках Android-разработки
\end{itemize}

\textbf{Здесь нужен скриншот работающего приложения с основными экранами}

Приложение соответствует всем минимальным требованиям и готово к дальнейшему расширению функциональности.

\section{Руководство пользователя}

Часть 2

\subsection{Пример перечисления}

\textbf{Android:}
\begin{itemize}
\item Android Studio или IntelliJ IDEA, последней стабильной версии.
\item Java Development Kit (JDK) 11.
\item Android SDK.
\end{itemize}

\textbf{Desktop (Kotlin/JVM):}
\begin{itemize}
\item JDK 11 или новее.
\item Android Studio или IntelliJ IDEA.
\end{itemize}

\textbf{iOS:}
\begin{itemize}
\item Xcode 14.3 или новее. Требуется операционная система macOS Ventura 13.0 или новее.
\end{itemize}

\textbf{Общие требования:}
\begin{itemize}
\item Kotlin Multiplatform Mobile (KMM) Plugin для Android Studio или IntelliJ IDEA.
\item Gradle (версия, указанная в проекте KMM).
\end{itemize}

\textbf{Примечание:}
При разработке для Android и iOS на Windows или Linux, вам может потребоваться использовать решения виртуализации для запуска и тестирования iOS приложений, такие как MacStadium или GitHub Actions.

\textbf{Примечание 2:}
Работоспособность Desktop приложения проверялась только на Windows и macOS. Возможность запуска приложения на Linux --- под вопросом.

\subsection{Итог запуска И ПРИМЕР КАРТИНКИ}

На рисунке~\ref{fig:run} приведен результат запуска приложения на платформах iOS, Android и Desktop соответственно.

\begin{figure}[!htb]
\centering
  \includegraphics[width=0.5\textwidth]{Blanks-4LaTeX-main/run.png}
\caption{Результат запуска на iOS, Android, Desktop}
\label{fig:run}
\end{figure}

\anonsection{ЗАКЛЮЧЕНИЕ} 

Анонимное заключение


\renewcommand\refname{СПИСОК ИСПОЛЬЗОВАННЫХ ИСТОЧНИКОВ}
% Список литературы
\clearpage
%\bibliographystyle{ugost2008s}  %utf8gost71u.bst} %utf8gost705u} %gost2008s}
{\catcode`"\active\def"{\relax}
\addcontentsline{toc}{section}{\protect\numberline{}\refname}%
%\bibliography{biblio} %здесь ничего не меняем, кроме, возможно, имени bib-файла
\printbibliography
}
\newpage
\settocdepth{section}

\anonsection{ПРИЛОЖЕНИЕ А}\label{sect:appl-a}

\vspace{-20pt}

\begin{longlisting}
\caption{Файл build.gradle.kts файл umbrella модуля.}
\label{lst:umbrella-build-gradle}
\begin{minted}[frame=single,fontsize = \footnotesize, linenos, xleftmargin = 1.5em,breaklines]{kotlin}
plugins {
    id("mp-compose-setup")
    id("android-setup")
    kotlin("native.cocoapods")
}
version = "0.0.1"
kotlin {
    cocoapods {
        summary = "MPP iOS SDK + Compose"
        homepage = "https://google.com"
        ios.deploymentTarget = "14.0"
        podfile = project.file("../../apps/iosApp/Podfile")
        framework {
            transitiveExport = false
            isStatic = true
            baseName = "shared_compose"
            freeCompilerArgs += "-Xbinary=bundleId=ru.andvl.mppapp"
            export(projects.common.core)
            export(projects.common.coreUtils)
            export(projects.common.coreCompose)
            export(projects.common.auth.data)
            export(projects.common.main.compose)
            export(projects.common.list.data)
            export(projects.common.list.compose)
            export(projects.common.umbrellaCore)
        }
    }
    sourceSets {
        commonMain {
            dependencies {
                implementation(projects.common.core)
                implementation(projects.common.coreUtils)
                implementation(projects.common.coreCompose)

                implementation(projects.common.auth.data)

                implementation(projects.common.main.compose)

                implementation(projects.common.list.data)
                implementation(projects.common.list.compose)

                implementation(projects.common.umbrellaCore)

                implementation(libs.kViewModel.core)
                implementation(libs.kViewModel.compose)
                implementation(libs.kViewModel.odyssey)

                implementation(libs.odyssey.core)
                implementation(libs.odyssey.compose)
            }
        }
        androidMain {
            dependencies {
                implementation(project.dependencies.platform(
                    libs.androidx.compose.bom
                ))
                implementation(libs.compose.activity)
            }
        }
        iosMain {
            dependencies {
                api(projects.common.core)
                api(projects.common.coreUtils)
                api(projects.common.coreCompose)
                api(projects.common.auth.data)
                api(projects.common.main.compose)
                api(projects.common.list.data)
                api(projects.common.list.compose)
                api(projects.common.umbrellaCore)
            }
        }
    }
}
\end{minted}
\end{longlisting}

\newpage
\settocdepth{section}

\anonsection{ПРИЛОЖЕНИЕ В}\label{sect:appl-v}

\vspace{-20pt}

\begin{figure}[!htb]
\centering
  \includegraphics[width=0.5\textwidth]{Blanks-4LaTeX-main/structure.png}
\caption{Структура проекта.}
\label{fig:project-structure}
\end{figure}

\end{document}

